\documentclass[12pt,a4paper]{article}

% OPTIONS FOR DRAFT
%\usepackage[light]{draftcopy}

% EXTERNAL PACKAGES
\usepackage{amsfonts}
\usepackage{amsmath}
%\usepackage{graphics}
%\usepackage[normal]{subfigure}
%\usepackage{theorem}
%\usepackage{fancyhdr}
%\usepackage{QED}
\usepackage{palatino}

% PAGE LAYOUT OPTIONS
\renewcommand{\baselinestretch}{1.3}
\hoffset=1.46cm
\oddsidemargin=0cm
\textwidth=14.5cm
\voffset=-1.54cm
\topmargin=0cm
\headheight=.75cm
\headsep=.75cm
\textheight=25cm
\footskip=1cm
%\pagestyle{fancy}
%\rhead{\thepage}
%\cfoot{}

% USER COMMANDS
%\newcommand{\eg}{eg.\ }
%\newcommand{\etc}{etc.\ }
%\newcommand{\ie}{ie.\ }
%\renewcommand{\div}{\operatorname{div}}
%\newcommand{\dee}{\operatorname{d}}

% THEOREM ENVIRONMENTS (needs the ``theorem'' package)
%\newtheorem{theorem}{Theorem}[section]
%{\theorembodyfont{\normalfont} \newtheorem{definition}{Definition}[section]}
%\newtheorem{axiom}[theorem]{Axiom}
%\newtheorem{lemma}[theorem]{Lemma}
%\newtheorem{corollary}[theorem]{Corollary}
%{\theorembodyfont{\normalfont} \newtheorem{example}{Example}[section]}
%{\theorembodyfont{\normalfont} \newtheorem{remark}{Remark}[section]}

%
%
% BEGIN
\begin{document}
  \begin{center}{\Huge Rotation} \\ {\Large Rado Faleti\v{c} }\end{center}

  %
  %
  % 2-DIMENSIONS
  \section{2-dimensions}

  Given a vector $\left( x,y \right) = \left( r\cos\theta , r\sin\theta \right) \in \mathbb{R}^2$, we can rotate this vector through an angle $\theta'$ by applying the following matrix:\[
  \begin{bmatrix} x' \\ y' \end{bmatrix} = 
  \begin{bmatrix} \cos\theta' & -\sin\theta' \\ \sin\theta' & \cos\theta' \end{bmatrix} \begin{bmatrix} x \\ y \end{bmatrix}
  \]
    {\sc Proof:}
    \begin{alignat*}{3}
      x' &= r \cos\left( \theta + \theta' \right) &\qquad y' &= r \sin\left( \theta + \theta' \right) \\
      &= r \left( \cos\theta \cos\theta' - \sin\theta \sin\theta' \right) &\qquad &= r \left( \cos\theta \sin\theta' + \sin\theta \cos\theta' \right) \\
      &= r \cos\theta \cos\theta' - r \sin\theta \sin\theta' &\qquad &= r \cos\theta \sin\theta' + r \sin\theta \cos\theta' \\
      &= x \cos\theta' - y \sin\theta' &\qquad &= x \sin\theta' + y \cos\theta'
    \end{alignat*}

    \pagebreak
    %
    %
    % 3-DIMENSIONS
    \section{3-dimensions}
    Given a vector $\left( x,y,z \right) = \left( r \sin\phi \cos\theta , r \sin\phi \sin\theta , r \cos\phi \right) \in \mathbb{R}^3$, we can rotate this vector through angles $\theta'$ and $\phi'$ by applying the following non-linear transformation:
    \[
    \begin{bmatrix} x' \\ y' \\ z' \end{bmatrix} =
    \begin{bmatrix}
      x \cos\phi' \cos\theta' - y \cos\phi' \sin\theta' + z \sin\phi' \left( \cos\theta \cos\theta' - \sin\theta \sin\theta' \right) \\
      x \cos\phi' \sin\theta' + y \cos\phi' \cos\theta' + z \sin\phi' \left( \cos\theta \sin\theta' + \sin\theta \cos\theta' \right) \\
      z \cos\phi' - r \sin\phi \sin\phi' 
    \end{bmatrix}
    \]
      {\sc Proof:}
      \begin{alignat*}{1}
	x' &= r \sin\left( \phi + \phi' \right) \cos\left( \theta + \theta' \right) \\
	&= r \left( \sin\phi \cos\phi' + \cos\phi \sin\phi' \right) \left( \cos\theta \cos\theta' - \sin\theta \sin\theta' \right) \\
	&= r \sin\phi \cos\phi' \cos\theta \cos\theta' - r \sin\phi \cos\phi' \sin\theta \sin\theta' \\
	&\qquad + r \cos\phi \sin\phi' \cos\theta \cos\theta' - r \cos\phi \sin\phi' \sin\theta \sin\theta' \\
	&= x \cos\phi' \cos\theta' - y \cos\phi' \sin\theta' + z \sin\phi' \left( \cos\theta \cos\theta' - \sin\theta \sin\theta' \right) \\
	y' &= r \sin\left( \phi + \phi' \right) \sin \left( \theta + \theta' \right) \\
	&= r \left( \sin\phi \cos\phi' + \cos\phi \sin\phi' \right) \left( \sin\theta \cos\theta' + \cos\theta \sin\theta' \right) \\
	&= r \sin\phi \cos\phi' \sin\theta \cos\theta' + r \sin\phi \cos\phi' \cos\theta \sin\theta' \\
	&\qquad + r \cos\phi \sin\phi' \sin\theta \cos\theta' + r \cos\phi \sin\phi' \cos\theta \sin\theta' \\
	&= y \cos\phi' \cos\theta' + x\cos\theta' \sin\theta' + z\sin\theta' \left( \sin\theta \cos\theta' + \cos\theta\sin\theta' \right) \\
	&= x\cos\theta' \sin\theta' + y \cos\phi' \cos\theta' + z\sin\theta' \left( \cos\theta\sin\theta' + \sin\theta \cos\theta' \right) \\
	z' &= r \cos \left( \phi + \phi' \right) \\
	&= r\cos\phi\cos\phi' - r\sin\phi\sin\phi' \\
	&= z \cos\phi' - r\sin\phi\sin\phi'
      \end{alignat*}

\end{document}
