\documentclass[12pt,a4paper]{article}

% OPTIONS FOR DRAFT
%\usepackage[light]{draftcopy}

% EXTERNAL PACKAGES
\usepackage{amsfonts}
\usepackage{amsmath}
\usepackage{epic}
\usepackage{eepic}
%\usepackage[normal]{subfigure}
%\usepackage{theorem}
%\usepackage{fancyhdr}
\usepackage{palatino}

% PAGE LAYOUT OPTIONS
\renewcommand{\baselinestretch}{1.3}
\hoffset=1.46cm
\oddsidemargin=0cm
\textwidth=14.5cm
\voffset=-1.54cm
\topmargin=0cm
\headheight=.75cm
\headsep=.75cm
\textheight=25cm
\footskip=1cm
%\pagestyle{fancy}
%\rhead{\thepage}
%\cfoot{}

% THEOREM ENVIRONMENTS (needs the ``theorem'' package)
%{\theorembodyfont{\normalfont} \newtheorem{definition}{Definition}[section]}

% USER COMMANDS
\newcommand{\eg}{eg.\ }
\newcommand{\ie}{ie.\ }
\newcommand{\dee}{\operatorname{d}}
%
%
% BEGIN
\begin{document}
  \begin{center}{\Huge Shock tube considerations}\end{center}
  \begin{itemize}
  \item The resevoir conditions of the shock tube are defined by: the total pressure, $P_0$, of the resevoir; the fill pressure, $P$; and the shock wave velocity, $\mathbf{v}$.
    
  \item The total temperature, $T_0$, can be calculated by entering $P_0$ into the Equilibrium Shock Tube Calculator (ESTC).
    
  \item The exit conditions of the tube are then given by entering $T_0$ and $P_0$ into STUBE.
  \end{itemize}

  The compression ratio, $\lambda$, of the compression tube is given by:
  \begin{equation*}
    \lambda = \frac{\text{volume of compression tube before compression}}{\text{volume of compression tube after compression}}
  \end{equation*}
  \begin{center}
    \setlength{\unitlength}{15pt}
    \begin{picture}(13,5)
      \put(5.75,3.25){\small $V_\text{i}$}
      % tube walls
      \path(0,5)(10,5)(10,4)(12.5,4)
      \path(0,2)(10,2)(10,3)(12.5,3)
      % diaphragm
      \path(9.9,4.9)(9.9,2.1)
      \put(9.9,0.7){\vector(0,1){1.2}}
      \put(8.1,0.2){\small diaphragm}
      % piston
      \path(1,4.9)(2.5,4.9)(2.5,2.1)(1,2.1)(1,2.6)(2,2.6)(2,4.4)(1,4.4)(1,4.9)
      \put(1.75,0.7){\vector(0,1){1.2}}
      \put(0.75,0.2){\small piston}
    \end{picture}
  \end{center}
  \begin{center}
    \setlength{\unitlength}{15pt}
    \begin{picture}(13,5)
      \put(8.75,3.25){\small $V_\text{f}$}
      % tube walls
      \path(0,5)(10,5)(10,4)(12.5,4)
      \path(0,2)(10,2)(10,3)(12.5,3)
      % diaphragm
      \path(9.9,4.9)(9.9,3.9)(10.3,3.8)
      \path(10.3,3.2)(9.9,3.1)(9.9,2.1)
      \put(9.9,0.7){\vector(0,1){1.2}}
      \put(8.1,0.2){\small diaphragm}
      % piston
      \path(6.5,4.9)(8,4.9)(8,2.1)(6.5,2.1)(6.5,2.6)(7.5,2.6)(7.5,4.4)(6.5,4.4)(6.5,4.9)
      \put(5.75,0.7){\vector(1,1){1.2}}
      \put(4.75,0.2){\small piston}
    \end{picture}
  \end{center}
  \ie $\lambda = \frac{V_\text{i}}{V_\text{f}} =  \frac{\rho_\text{f}}{\rho_\text{i}} = \left( \frac{P_\text{f}}{P_\text{i}} \right)^{\frac{1}{\gamma}} = \left( \frac{D}{C} \right)^{\frac{1}{\gamma}}$, where $P_\text{f}$ ($=D$) is the diaphragm burst pressure and $P_\text{i}$ ($=C$) is the compression tube fill pressure.

  The empirical equation for the free piston shock tube (FPST) is
  \begin{equation*}
    \frac{R\lambda}{D} = \text{``constant'',}
  \end{equation*}
  where $R$ is the resevoir fill pressure.

\end{document}
