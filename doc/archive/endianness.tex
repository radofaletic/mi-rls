\documentclass[12pt,a4paper]{article}

% OPTIONS FOR DRAFT
%\usepackage[light]{draftcopy}

% EXTERNAL PACKAGES
%\usepackage{amsfonts}
%\usepackage{amsmath}
%\usepackage[normal]{subfigure}
%\usepackage{theorem}
%\usepackage{fancyhdr}
\usepackage{palatino}

% PAGE LAYOUT OPTIONS
\renewcommand{\baselinestretch}{1.3}
\hoffset=1.46cm
\oddsidemargin=0cm
\textwidth=14.5cm
\voffset=-1.54cm
\topmargin=0cm
\headheight=.75cm
\headsep=.75cm
\textheight=25cm
\footskip=1cm
%\pagestyle{fancy}
%\rhead{\thepage}
%\cfoot{}

% THEOREM ENVIRONMENTS (needs the ``theorem'' package)
%{\theorembodyfont{\normalfont} \newtheorem{definition}{Definition}[section]}

% USER COMMANDS
\newcommand{\eg}{eg.\ }
\newcommand{\ie}{ie.\ }
\newcommand{\dee}{\operatorname{d}}
%
%
% BEGIN
\begin{document}
  \begin{center}{\Huge Endian'ness}\end{center}
  Sun Workshop Developer Products: Data\newline
  http://suncom.bilkent.edu.tr/workshop/wp-archdiff/data.html
  \begin{center}
    \begin{tabular}{l|l}
      platform	& big/little	\\ \hline\hline
      x86	&	little	\\ \hline
      SPARC	&	big	\\ \hline
      HP	&	big	\\ \hline
      DEC	&	little	\\ \hline
      SGI	&	big	\\ \hline
      Mac	&	big	\\ \hline
    \end{tabular}
  \end{center}
  Note: the difference between big and little endianness is the ordering of the bytes, not the bits. Consider ``\texttt{int i=1546;}''. Little endian will be ordered as:
  \begin{center}
    \begin{tabular}{c|c|c|c}
      \texttt{n+3}      & \texttt{n+2}      & \texttt{n+1}      & \texttt{n} \\ \hline
      \texttt{00000000} & \texttt{00000000} & \texttt{00000110} & \texttt{00001010}
    \end{tabular}
  \end{center}
  and referenced as ``\texttt{\&i=n}''. Big endian will store this value with the bytes reversed, \ie reverse the table.

\end{document}
