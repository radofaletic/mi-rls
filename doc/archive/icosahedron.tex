\documentclass[12pt,a4paper]{article}

% OPTIONS FOR DRAFT
%\usepackage[light]{draftcopy}

% EXTERNAL PACKAGES
\usepackage{amsfonts}
\usepackage{amsmath}
\usepackage{epic}
\usepackage{eepic}
%\usepackage[normal]{subfigure}
\usepackage{theorem}
%\usepackage{fancyhdr}
%\usepackage{QED}
%\usepackage{times}
%\usepackage{mathtime}

% PAGE LAYOUT OPTIONS
\renewcommand{\baselinestretch}{1.3}
\hoffset=1.46cm
\oddsidemargin=0cm
\textwidth=14.5cm
\voffset=-1.54cm
\topmargin=0cm
\headheight=.75cm
\headsep=.75cm
\textheight=25cm
\footskip=1cm
%\pagestyle{fancy}
%\rhead{\thepage}
%\cfoot{}

% USER COMMANDS
%\newcommand{\eg}{eg.\ }
%\newcommand{\etc}{etc.\ }
%\newcommand{\ie}{ie.\ }
%\renewcommand{\div}{\operatorname{div}}
%\newcommand{\dee}{\operatorname{d}}

% THEOREM ENVIRONMENTS (needs the ``theorem'' package)
\newtheorem{theorem}{Theorem}
%{\theorembodyfont{\normalfont} \newtheorem{definition}{Definition}[section]}
%\newtheorem{axiom}[theorem]{Axiom}
%\newtheorem{lemma}[theorem]{Lemma}
%\newtheorem{corollary}[theorem]{Corollary}
%{\theorembodyfont{\normalfont} \newtheorem{example}{Example}[section]}
%{\theorembodyfont{\normalfont} \newtheorem{remark}{Remark}[section]}

%
%
% BEGIN
\begin{document}
  \begin{center}{\Huge the Icosahedron}\end{center}

  \begin{theorem}
    If an icosahedron is placed inside a sphere of unit radius then the triangles subtended from both poles have an angular width, at their base, of $\frac{2\pi}{5}$ and the bases are an angular distance of $\phi$ from the pole, where $\phi$ is given by $\cos\phi = \frac{\cos\frac{2\pi}{5}}{1-\cos\frac{2\pi}{5}}$.
  \end{theorem}

  PROOF:\\
  \begin{center}
    \setlength{\unitlength}{15pt}
    \begin{picture}(21,21)
      % graph axes
      \put(19,14.5){\vector(-2,-1){18}}
      \put(0.5,5.3){\small $x$}
      \put(0,10){\vector(1,0){20}}
      \put(20.1,9.8){\small $y$}
      \put(10,0){\vector(0,1){20}}
      \put(9.8,20.1){\small $z$}
      % sphere
      \put(10,10){\circle{18}}
      % spherical triangle
      \put(17.5,9){\arc{25.1}{3.14}{4.07}}
      \put(0,10.6){\arc{26.2}{5.58}{6.4}}
      \put(9,20){\arc{23.41}{1.22}{1.92}}
      \put(10,19){\makebox(0,0){$\bullet$}}
      \put(5,9){\makebox(0,0){$\bullet$}}
      \put(13,9){\makebox(0,0){$\bullet$}}
      % drop lines
      \dashline{0.25}(5,9)(5,7.5)(13,7.5)(13,9)(10,10)(5,9)(10,11.5)
      \dashline{0.25}(10,10)(13,7.5)
      \path(9.5,11.25)(9.5,11.75)(10,12)
      \put(7.5,10.7){\small $l$}
      % planar triangle
      \dottedline{0.1}(10,19)(5,9)(13,9)(10,19)
      \put(9,8.5){\small $t$}
      \put(11.5,14){\small $t$}
      \put(7.3,14){\small $t$}
      % arc
      \put(10,10){\arc{2}{2.94}{4.71}}
      \put(8.8,10.5){\small $\phi$}
    \end{picture}
  \end{center}
  Firstly, there are five triangles around each pole, so the angular width of each triangle must be $\frac{2\pi}{5}$. Let $t$ be the length of the straight line connecting two nodes of a triangle. Then, by the cosine rule,
  \begin{equation*}
    t^2 = l^2 + l^2 - 2l^2 \cos\frac{2\pi}{5} = 2l^2\left( 1 - \cos\frac{2\pi}{5} \right).
  \end{equation*}
  Since the sphere has unit radius we have $l=\sin\phi$ and $h=1-\cos\phi$. Pythagoras' theorem gives that
  \begin{alignat*}{1}
    &\qquad t^2 = l^2 + h^2 \\
    &\Rightarrow 2l^2 \left( 1-\cos\frac{2\pi}{5} \right) = l^2 + h^2 \\
    &\Rightarrow 2\sin^2\phi \left( 1-\cos\frac{2\pi}{5} \right) = \sin^2\phi + \left( 1-\cos\phi \right)^2 \\
    &\Rightarrow 2\sin^2\phi \left( 1-\cos\frac{2\pi}{5} \right) = \sin^2\phi + 1 - 2\cos\phi + \cos^2\phi \\
    &\Rightarrow 2\sin^2\phi \left( 1-\cos\frac{2\pi}{5} \right) = 2\left( 1-\cos\phi \right) \\
    &\Rightarrow \left( 1-\cos^2\phi \right) \left( 1 - \cos\frac{2\pi}{5} \right) = \left( 1-\cos\phi \right) \\
    &\Rightarrow \left( 1-\cos\phi \right) \left( 1+\cos\phi \right) \left( 1 - \cos\frac{2\pi}{5} \right) = \left( 1-\cos\phi \right) \\
    &\Rightarrow 1+\cos\phi = \frac{1}{1-\cos\frac{2\pi}{5}} \\
    &\Rightarrow \cos\phi = \frac{\cos\frac{2\pi}{5}}{1-\cos\frac{2\pi}{5}}.
  \end{alignat*}

  \begin{center}
    spherical co-ordinates for points on the icosahedron\\
    \begin{tabular}{|c||c|c|} \hline
      \    & $\qquad\theta\qquad$ & $\qquad\phi\qquad$ \\ \hline\hline
      $1$  & $0$                  & $0$ \\ \hline
      $2$  & $0$                  & $\arccos\left(\frac{\cos\frac{2\pi}{5}}{1-\cos\frac{2\pi}{5}}\right)$ \\ \hline
      $3$  & $\frac{2\pi}{5}$     & $\arccos\left(\frac{\cos\frac{2\pi}{5}}{1-\cos\frac{2\pi}{5}}\right)$ \\ \hline
      $4$  & $\frac{4\pi}{5}$     & $\arccos\left(\frac{\cos\frac{2\pi}{5}}{1-\cos\frac{2\pi}{5}}\right)$ \\ \hline
      $5$  & $\frac{6\pi}{5}$     & $\arccos\left(\frac{\cos\frac{2\pi}{5}}{1-\cos\frac{2\pi}{5}}\right)$ \\ \hline
      $6$  & $\frac{8\pi}{5}$     & $\arccos\left(\frac{\cos\frac{2\pi}{5}}{1-\cos\frac{2\pi}{5}}\right)$ \\ \hline
      $7$  & $\frac{9\pi}{5}$     & $\pi - \arccos\left(\frac{\cos\frac{2\pi}{5}}{1-\cos\frac{2\pi}{5}}\right)$ \\ \hline
      $8$  & $\frac{\pi}{5}$      & $\pi - \arccos\left(\frac{\cos\frac{2\pi}{5}}{1-\cos\frac{2\pi}{5}}\right)$ \\ \hline
      $9$  & $\frac{3\pi}{5}$     & $\pi - \arccos\left(\frac{\cos\frac{2\pi}{5}}{1-\cos\frac{2\pi}{5}}\right)$ \\ \hline
      $10$ & $\pi$                & $\pi - \arccos\left(\frac{\cos\frac{2\pi}{5}}{1-\cos\frac{2\pi}{5}}\right)$ \\ \hline
      $11$ & $\frac{7\pi}{5}$     & $\pi - \arccos\left(\frac{\cos\frac{2\pi}{5}}{1-\cos\frac{2\pi}{5}}\right)$ \\ \hline
      $12$ & $0$                  & $\pi$ \\ \hline
    \end{tabular}
  \end{center}

\end{document}
