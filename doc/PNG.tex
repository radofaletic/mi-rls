\documentclass[12pt,a4paper]{article}

% OPTIONS FOR DRAFT
%\usepackage[light]{draftcopy}

% EXTERNAL PACKAGES
\usepackage{amsfonts}
\usepackage{amsmath}
%\usepackage{epic}
%\usepackage{eepic}
%\usepackage[normal]{subfigure}
\usepackage{theorem}
%\usepackage{fancyhdr}
%\usepackage{palatino}
\usepackage{hyperref}

% PAGE LAYOUT OPTIONS
%\renewcommand{\baselinestretch}{1.3}
\hoffset=1.46cm
\oddsidemargin=0cm
\textwidth=14.5cm
\voffset=-1.54cm
\topmargin=0cm
\headheight=.75cm
\headsep=.75cm
\textheight=25cm
\footskip=1cm
%\pagestyle{fancy}
%\rhead{\thepage}
%\cfoot{}

% USER COMMANDS
\newcommand{\eg}{eg.\ }
\newcommand{\etc}{etc.\ }
\newcommand{\ie}{ie.\ }
\renewcommand{\div}{\operatorname{div}}
\newcommand{\dee}{\operatorname{d}}

% THEOREM ENVIRONMENTS (needs the ``theorem'' package)
%\newtheorem{theorem}{Theorem}[section]
%{\theorembodyfont{\normalfont} \newtheorem{definition}{Definition}[section]}
%\newtheorem{axiom}[theorem]{Axiom}
%\newtheorem{lemma}[theorem]{Lemma}
%\newtheorem{corollary}[theorem]{Corollary}
%{\theorembodyfont{\normalfont} \newtheorem{example}{Example}[section]}
%{\theorembodyfont{\normalfont} \newtheorem{remark}{Remark}[section]}


%
%
% BEGIN
\begin{document}

  \section{PNG file format}
  All two dimensional images and data that are produced at various stage of the computation are stored in PNG image file format~\cite{PNG}. The reason for this is to allow easy, portable viewing of the results, since PNG files are viewable in every modern web browser and image viewer. PNG is a free image format, available to everyone to use as they see fit, plus free software libraries are available for the reading and writing of PNG files~\cite{libpng}.

  Different pieces of information in a PNG file are called chunks. For example, there is a chunk for describing the images transparency, tRNS, and another chunk for describing the creation time, tIME.

  One benefit of the PNG file format is that it allows unlimited storage of textual chunks, tEXT. By appropriately choosing and defining new tEXT chunks we can completely define a floating point data set by storing it in standard floating point format (\eg 2.45654e-12) in a tEXT chunk. The displayable part of the PNG image can then contain an integer representation of the floating point data for easy viewing purposes.

  To be able to store a series of images in one PNG file, an image was created that contains a grid of the series. For example, a series of 5 images would be stored in a $2\times 3$ fashion, with each image being separated by single transparent pixel. To be able to describe this configuration, a tEXT chunk is defined, called \emph{collage}, that stores the number of images and the number of rows and columns in which they are layed out. For example, the \emph{collage} string may contain ``17:4,5'', which means that 17 images are represented on a grid with 4 rows and 5 columns:
  \begin{center}
    \setlength{\unitlength}{3pt}
    \begin{picture}(104,43)
      \put(0,33){\framebox(20,10){\small image~1}}
      \put(21,33){\framebox(20,10){\small image~2}}
      \put(42,33){\framebox(20,10){\small image~3}}
      \put(63,33){\framebox(20,10){\small image~4}}
      \put(84,33){\framebox(20,10){\small image~5}}
      \put(0,22){\framebox(20,10){\small image~6}}
      \put(21,22){\framebox(20,10){\small image~7}}
      \put(42,22){\framebox(20,10){\small image~8}}
      \put(63,22){\framebox(20,10){\small image~9}}
      \put(84,22){\framebox(20,10){\small image~10}}
      \put(0,11){\framebox(20,10){\small image~11}}
      \put(21,11){\framebox(20,10){\small image~12}}
      \put(42,11){\framebox(20,10){\small image~13}}
      \put(63,11){\framebox(20,10){\small image~14}}
      \put(84,11){\framebox(20,10){\small image~15}}
      \put(0,0){\framebox(20,10){\small image~16}}
      \put(21,0){\framebox(20,10){\small image~17}}
    \end{picture}
  \end{center}
  Since this project is concerned with tomography, there have been tEXT chunks defined to describe the angles of projection. The first of these is called \emph{raxis}, which describes whether the images have been rotated about the x-axis or y-axis. The possible stored values are $\text{X}$ or $\text{Y}$. The second tEXT chunk concerned with projections is simply called \emph{angles}, which is a series of comma separated floating-point numbers that give the angle of rotation for each image. The angles are stored as degrees, as opposed to radians. A value of ``$\text{default}$'' here means that the angles are to be equi-spaced between $0^\circ$ and $180^\circ$. So, for the example above, the angle for image $i$ would be:
  \[
  \left( i-1 \right) \frac{180}{\text{no. images}} = \left( i - 1\right) \frac{180}{17} \text{.}
  \]

  
  % BIBLIOGRAPHY
  \pagebreak
  \bibliographystyle{plain}
  \bibliography{computing}
\end{document}
