\documentclass[12pt,a4paper]{article}

% OPTIONS FOR DRAFT
%\usepackage[light]{draftcopy}

% EXTERNAL PACKAGES
\usepackage{makeidx}
\usepackage{amsfonts}
\usepackage{amsmath}
\usepackage{epic}
\usepackage{eepic}
%\usepackage[normal]{subfigure}
\usepackage{theorem}
%\usepackage{fancyhdr}
\usepackage{palatino}

% PAGE LAYOUT OPTIONS
\renewcommand{\baselinestretch}{1.3}
\hoffset=1.46cm
\oddsidemargin=0cm
\textwidth=14.5cm
\voffset=-1.54cm
\topmargin=0cm
\headheight=.75cm
\headsep=.75cm
\textheight=25cm
\footskip=1cm
%\pagestyle{fancy}
%\rhead{\thepage}
%\cfoot{}

% USER COMMANDS
\newcommand{\eg}{eg.\ }
\newcommand{\etc}{etc.\ }
\newcommand{\ie}{ie.\ }
\renewcommand{\div}{\operatorname{div}}
\newcommand{\dee}{\operatorname{d}}

% THEOREM ENVIRONMENTS (needs the ``theorem'' package)
\newtheorem{theorem}{Theorem}[section]
{\theorembodyfont{\normalfont} \newtheorem{definition}{Definition}[section]}
%\newtheorem{axiom}[theorem]{Axiom}
%\newtheorem{lemma}[theorem]{Lemma}
\newtheorem{corollary}[theorem]{Corollary}
%{\theorembodyfont{\normalfont} \newtheorem{example}{Example}[section]}
%{\theorembodyfont{\normalfont} \newtheorem{remark}{Remark}[section]}

% INDEX
\makeindex
\index{transform!Fourier|see{Fourier transform}}

%
%
% BEGIN
\begin{document}
  \section{Fourier}

  %
  % THE FOURIER TRANSFORM %
  \index{Fourier transform}

  \begin{definition}[Fourier transform]
    The \emph{Fourier transform} of a function $f(\mathbf{x})$ is defined by:
    \begin{equation}
      (\mathcal{F}f)(\boldsymbol{\nu}) = F(\boldsymbol{\nu}) = \int_{\mathbb{R}^n} f(\mathbf{x}) e^{-i2\pi \boldsymbol{\nu}\cdot \mathbf{x}} \, \dee \mathbf{x} \text{.}
    \end{equation}
  \end{definition}

  \begin{theorem}[Inverse Fourier transform]
    The \emph{inverse Fourier transform} is a function $\mathcal{F}^{-1}$ that satisfies $\mathcal{F}^{-1} \left( \mathcal{F} f \right) = f $ for any function $f$. The function $\mathcal{F}^{-1}$ of a function $F ( \boldsymbol{\nu} )$ is given by:
    \begin{equation}
      (\mathcal{F}^{-1}F)(\mathbf{x}) = f(\mathbf{x}) = \int_{\mathbb{R}^n} F(\boldsymbol{\nu}) e^{i2\pi \mathbf{x} \cdot \boldsymbol{\nu}} \, \dee \boldsymbol{\nu} \text{.}
    \end{equation}
  \end{theorem}

  \begin{theorem}[Fourier slice theorem]\index{Fourier transform!slice theorem}
    Suppose $T_{\boldsymbol{\omega}} \left( r \right)$ is a tomogram of a function $f$. Then:
    \begin{equation}
      \left( \mathcal{F}_1 T_{\boldsymbol{\omega}}\right) \left( r \right) = \left( \mathcal{F}_n f \right) \left( r \boldsymbol{\omega} \right) \text{.}
    \end{equation}
  \end{theorem}
  PROOF:
  \begin{eqnarray*}
    \left( \mathcal{F}_1 T_{\boldsymbol{\omega}} \right) \left( r \right) & = & \int_{-\infty}^\infty \left( \int_{\mathbb{R}^n} f \left( \mathbf{z} \right) \delta \left( s - \mathbf{z} \cdot \boldsymbol{\omega} \right) \dee \mathbf{z} \right) e^{-2\pi i s r} \dee s \\
    & = & \int_{\mathbb{R}^n} f \left( \mathbf{z} \right) \left( \int_{-\infty}^\infty \delta \left( s - \mathbf{z} \cdot \boldsymbol{\omega} \right) e^{-2\pi i s r} \dee s \right) \dee \mathbf{z} \\
    & = & \int_{\mathbb{R}^n} f \left( \mathbf{z} \right) \left( e^{-2\pi i \mathbf{z} \cdot \boldsymbol{\omega} r} \left( \mathcal{F} \delta \right) \left( r \right) \right) \dee \mathbf{z} \\
    & = & \int_{\mathbb{R}^n} f \left( \mathbf{z} \right) e^{-2\pi i \mathbf{z} \cdot \left( r \boldsymbol{\omega} \right) } \dee \mathbf{z} \\
    & = & \left( \mathcal{F}_n f \right) \left( r \boldsymbol{\omega} \right)
  \end{eqnarray*}

  This simple theorem can then be used to recover the data function $f$ from its projections $p$.

  \section{DFT process}
  For the following, suppose that their are $n$ sample points in each projection.
  \begin{enumerate}
    \item For each projection, take the DFT.
    \item The DFT is not the same as the Fourier transform. So we need to do a shift. The first $\left( n/2 \right) + 1$ (the division is rounded down to the nearest integer) elements need to be moved/wrapped to the end of the array.\cite[pg~361]{bracewell}
    \item Each of these transformed projections need to be put into a 2D plane, according to the inital angle of projection.
    \item This 2D plane needs to be interpolated to create a set of values at Cartesian coordinates, as opposed to circular.
    \item Cycle over this data set and find which $x$ and $y$ coordinate, $\left( i, j \right)$, holds the largest value for the real component. This point corresponds to the DC component of the Fourier signal.
    \item Do the reverse 2D DFT.
    \item Again, the DFT is not the same at the Fourier transform. So we need to shift the points of the grid, firstly in the $-x$ direction so that the $i$ coordinate is shifted to $x=0$, and then secondly in the $-y$ direction so that the $j$ coordinate is shifted to $y=0$. Remember that this shift is to be wrapped.
    \item Now we have the solution.
  \end{enumerate}

  %
  % BIBLIOGRAPHY
  \pagebreak
  \bibliographystyle{plain}
  \bibliography{mathematics,seismology}
  \pagebreak
  \printindex
\end{document}
